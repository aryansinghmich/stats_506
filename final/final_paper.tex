%%
%% Copyright 2022 OXFORD UNIVERSITY PRESS
%%
%% This file is part of the 'oup-authoring-template Bundle'.
%% ---------------------------------------------
%%
%% It may be distributed under the conditions of the LaTeX Project Public
%% License, either version 1.2 of this license or (at your option) any
%% later version.  The latest version of this license is in
%%    http://www.latex-project.org/lppl.txt
%% and version 1.2 or later is part of all distributions of LaTeX
%% version 1999/12/01 or later.
%%
%% The list of all files belonging to the 'oup-authoring-template Bundle' is
%% given in the file `manifest.txt'.
%%
%% Template article for OXFORD UNIVERSITY PRESS's document class `oup-authoring-template'
%% with bibliographic references
%%

%%%CONTEMPORARY%%%
%\documentclass[unnumsec,webpdf,contemporary,large]{oup-authoring-template}%
\documentclass[unnumsec,webpdf,contemporary,large,namedate]{oup-authoring-template}% uncomment this line for author year citations and comment the above
%\documentclass[unnumsec,webpdf,contemporary,medium]{oup-authoring-template}
%\documentclass[unnumsec,webpdf,contemporary,small]{oup-authoring-template}

%%%MODERN%%%
%\documentclass[unnumsec,webpdf,modern,large]{oup-authoring-template}
%\documentclass[unnumsec,webpdf,modern,large,namedate]{oup-authoring-template}% uncomment this line for author year citations and comment the above
%\documentclass[unnumsec,webpdf,modern,medium]{oup-authoring-template}
%\documentclass[unnumsec,webpdf,modern,small]{oup-authoring-template}

%%%TRADITIONAL%%%
%\documentclass[unnumsec,webpdf,traditional,large]{oup-authoring-template}
%\documentclass[unnumsec,webpdf,traditional,large,namedate]{oup-authoring-template}% uncomment this line for author year citations and comment the above
%\documentclass[unnumsec,namedate,webpdf,traditional,medium]{oup-authoring-template}
%\documentclass[namedate,webpdf,traditional,small]{oup-authoring-template}

%\onecolumn % for one column layouts

%\usepackage{showframe}

\graphicspath{{Fig/}}

% line numbers
%\usepackage[mathlines, switch]{lineno}
%\usepackage[right]{lineno}

\theoremstyle{thmstyleone}%
\newtheorem{theorem}{Theorem}%  meant for continuous numbers
%%\newtheorem{theorem}{Theorem}[section]% meant for sectionwise numbers
%% optional argument [theorem] produces theorem numbering sequence instead of independent numbers for Proposition
\newtheorem{proposition}[theorem]{Proposition}%
%%\newtheorem{proposition}{Proposition}% to get separate numbers for theorem and proposition etc.
\theoremstyle{thmstyletwo}%
\newtheorem{example}{Example}%
\newtheorem{remark}{Remark}%
\theoremstyle{thmstylethree}%
\newtheorem{definition}{Definition}

\usepackage[hyphens]{url}
\usepackage{hyperref}
\def\UrlBreaks{\do\/\do-}

\begin{document}

\journaltitle{University of Michigan}
\DOI{12.0125/bioinformatics/xxxxxx}
\copyrightyear{2025}
\pubyear{2025}
%\access{Advance Access Publication Date: Day Month Year}
\appnotes{STATS 506 Final Project}

\firstpage{1}

%\subtitle{Subject Section}

\title[Genomic Risk Stratification in Breast Cancer]{Genomic Risk Stratification of Breast Cancer Patients Using Penalized Survival Models}

% Spatially Informed Clustering Improves Marker-Based ell-Type Annotation in Visium Transcriptomics Analysis ofin Breast Cancer

\author[1,2,$\ast$]{Aryan Singh}


\authormark{Singh}

\address[1]{\orgdiv{Department of Statistics}, \orgname{University of Michigan}}
\address[2]{\orgdiv{Department of Computational Medicine \& Bioinformatics}, \orgname{Michigan Medicine, University of Michigan}}

\corresp[$\ast$]{Corresponding author. \href{email:aryansin@umich.edu}{aryansin@umich.edu}}

%\received{Date}{0}{Year}
%\revised{Date}{0}{Year}
%\accepted{Date}{0}{Year}

%\editor{Associate Editor: Name}

\abstract{
Gene expression data offer insight into molecular drivers of cancer prognosis, but high dimensionality and correlated predictors complicate survival modeling. Publicly available breast cancer microarray data from the GSE7390 cohort were analyzed to construct a genomic risk score for relapse-free survival using elastic net–regularized Cox proportional hazards modeling. After normalization and preprocessing, gene expression predictors were incorporated into a penalized survival framework with model tuning via cross-validation. Patients were stratified into high- and low-risk groups based on predicted risk scores. The model demonstrated good discriminative performance, with a concordance index of approximately 0.77, and Kaplan–Meier analysis showed clear separation between risk groups (log-rank $p < 0.0001$). These results highlight the utility of penalized survival models for genomic risk stratification. \\
\textbf{Availability:} Source code files are available at \url{https://github.com/aryansinghmich/stats_506}.\\
\textbf{Contact:} \ \href{mailto:aryansin@umich.edu}{aryansin@umich.edu} \\
\noindent\rule{\textwidth}{0.6pt}
}

\maketitle

\section{1 Introduction}\label{sec1}

Gene expression profiling has been critical to advances in prognostic modeling in breast cancer by enabling systematic investigation of molecular heterogeneity beyond traditional clinical factors \citep{perou2000molecular, sorlie2001gene}. A substantial body of work has shown that transcriptional patterns measured in primary tumors are associated with disease progression, treatment response, and long-term results, including relapse-free survival \citep{vantveer2002gene, paik2004multigene}. Despite these successes, translating high-dimensional gene expression data into reliable prognostic tools remains challenging. The number of measured predictors typically far exceeds the number of patients, and strong correlations among genes can lead to overfitting and unstable estimates when classical regression methods are applied directly \citep{tibshirani1997lasso, simon2011regularization}.

Survival analysis provides a natural framework for modeling time-to-event outcomes such as cancer relapse, with the Cox proportional hazards model remaining one of the most widely used approaches in biomedical research \citep{cox1972regression}. However, naively extending Cox models to genome-scale expression data is often impractical without additional constraints. Penalized regression methods, including the lasso and elastic net, address this limitation by shrinking coefficient estimates and performing variable selection, producing more stable and interpretable models in high-dimensional settings \citep{zou2005elasticnet, simon2011regularization}. As a result, these approaches have become increasingly common in genomic survival studies, where correlated expression patterns are the norm and interpretability remains an important goal.

Publicly available breast cancer expression datasets with annotated survival outcomes provide an opportunity to assess the practical performance of penalized survival models in a reproducible context \citep{wang2005gene, edgar2002geo}. By combining elastic net regularization with Cox proportional hazards modeling, gene expression profiles can be summarized into patient-level risk scores that support stratification into clinically relevant risk groups. The performance of the model can then be evaluated using concordance measures and Kaplan–Meier survival curves, providing a transparent link between molecular predictors and observed clinical outcomes.

\section{2 Methods}\label{sec2}

\subsection{2.1 Dataset}\label{subsec2_1}
Breast cancer gene expression data were obtained from the NCBI Gene Expression Omnibus (GEO) under accession GSE7390 \citep{wang2005gene, edgar2002geo}. The dataset consists of microarray-based expression profiles measured using the Affymetrix Human Genome U133A platform (GPL96) for primary breast tumor samples. Clinical annotations include relapse-free survival time and an event indicator denoting whether a relapse occurred or the observation was censored. After aligning expression and phenotype data, a total of 198 samples with complete survival information were retained for downstream analysis.

\subsection{2.2 Expression Data Preprocessing}\label{subsec2_2}
Raw expression values were extracted from the GEO series matrix file and examined to assess their scale. When necessary, expression intensities were transformed using a log$_2(x+1)$ transformation. To ensure comparability across samples, quantile normalization was applied using the \texttt{limma} package. The expression matrix was then transposed to form a samples-by-genes matrix suitable for regression modeling, and genes with near-zero variance across samples were removed to improve numerical stability.

\subsection{2.3 Survival Outcome Definition}\label{subsec2_3}
Relapse-free survival time was defined as the number of days from initial diagnosis to documented disease relapse or last follow-up. An event indicator distinguished relapse events from censored observations. These variables were combined into a survival object which formed the basis for all subsequent survival analyses.

\subsection{2.4 Penalized Cox Proportional Hazards Modeling}\label{subsec2_4}
Gene expression predictors were modeled using a Cox proportional hazards framework with elastic net regularization to address the high dimensionality and correlation structure of genomic features \citep{cox1972regression, tibshirani1997lasso, zou2005elasticnet}. Penalized estimation was performed by minimizing the negative partial log-likelihood with an elastic net penalty,
\begin{equation}
\mathcal{L}(\boldsymbol{\beta}) =
- \ell(\boldsymbol{\beta})
+ \lambda \left[
(1 - \alpha)\frac{1}{2} \lVert \boldsymbol{\beta} \rVert_2^2
+ \alpha \lVert \boldsymbol{\beta} \rVert_1
\right],
\end{equation}
where $\lambda$ controls the overall strength of regularization and $\alpha$ determines the balance between L1 and L2 penalties. Models were fit using the \texttt{glmnet} package with $\alpha = 0.5$.

The regularization parameter $\lambda$ was selected via 10-fold cross-validation with fixed fold assignments. To avoid degenerate solutions, the final model was chosen as the solution with minimal cross-validated error among models containing a nonzero set of coefficients.

\subsection{2.5 Risk Score Construction and Gene-Level Interpretation}\label{subsec2_5}
For each patient, a genomic risk score was computed as the linear predictor from the fitted elastic net Cox model. Model performance was evaluated using the concordance index (C-index), which measures agreement between predicted risk scores and observed survival times. Patients were stratified into high- and low-risk groups using a median risk split, and differences in relapse-free survival were visualized using Kaplan–Meier curves with log-rank testing.

Genes with nonzero coefficients in the final model were extracted and ranked by the magnitude of their estimated effects. Affymetrix probe identifiers were mapped to gene symbols using platform annotations, retaining a primary symbol when multiple mappings were present. Positive coefficients indicate increased relapse risk, whereas negative coefficients indicate protective associations.

\section{3 Results}\label{sec3}

\subsection{3.1 Exploratory structure of the expression data}\label{subsec3_1}
After quantile normalization, the GSE7390 expression values were on a log$_2$-like scale (range approximately $-2.86$ to $16.92$). PCA was applied to the top 2{,}000 most variable probes as a diagnostic visualization of sample to sample variation (Supplementary Figure S1). Relapse events and censored observations did not form clearly separable clusters in the first two principal components, indicating that relapse status is not driven by a single dominant axis of expression variance. This pattern suggests that prognostic signal is distributed across multiple genes rather than captured by a low-dimensional summary, motivating the use of a multivariate survival modeling approach.

\subsection{3.2 Elastic net Cox risk stratification}\label{subsec3_2}
Relapse-free survival outcomes were available for 198 patients, including 91 relapse events and 107 censored observations, with follow-up times ranging from 121 to 8{,}711 days. An elastic net Cox model ($\alpha = 0.5$) identified a parsimonious set of expression features associated with relapse risk, selecting 21 nonzero coefficients at the chosen penalty level ($\lambda^{\ast} = 0.3089$). This model yielded a continuous genomic risk score that varied substantially across patients, reflecting heterogeneity in predicted relapse risk.

When patients were stratified according to this score, Kaplan–Meier analysis revealed a pronounced divergence in relapse-free survival between groups (log-rank $p < 0.0001$; Figure~\ref{fig:km}). Individuals classified as high risk experienced markedly earlier relapse events compared to those in the low-risk group, while low-risk patients exhibited prolonged relapse-free survival. These findings indicate that the elastic net–derived risk score captures clinically meaningful prognostic variation that is not evident from unsupervised expression structure alone.

\begin{figure}[t]
\centering
\includegraphics[width=0.95\linewidth]{figure1_km.png}[-3mm]
\includegraphics[width=0.95\linewidth]{figure1_km_risktable.png}
\caption{Kaplan–Meier relapse-free survival curves stratified by the elastic net Cox genomic risk score, with numbers at risk shown below.}
\label{fig:km}
\end{figure}

\subsection{3.3 Gene level signals from the fitted model}\label{subsec3_3}
To characterize the molecular features underlying the observed risk stratification, the nonzero elastic net Cox coefficients at $\lambda^{\ast}$ were examined and mapped from Affymetrix probe identifiers to gene symbols. The selected genes included both positive and negative coefficients, corresponding to increased and decreased relapse risk, respectively. Figure~\ref{fig:coeffs} displays the top coefficients ranked by absolute magnitude, highlighting the strongest contributors to the genomic risk score.

Although the model was optimized for prediction rather than inference, the direction and relative size of these coefficients provide insight into expression patterns most strongly associated with prognosis in this cohort. Several genes exhibited substantial effects in either direction, underscoring the multigenic nature of relapse risk. 

Among the strongest contributors to the fitted model were SLC38A7, which exhibited a positive coefficient, and ANXA7, which showed a strong negative association with relapse risk. SLC38A7 encodes a sodium-coupled neutral amino acid transporter belonging to the solute carrier family, and its function in amino acid transport has been linked to cellular metabolism and nutrient uptake, processes that can support tumor growth and aggressiveness when upregulated \citep{slc38a7_info, slc38a7_cancer_association}. In contrast, ANXA7 (Annexin A7) is a calcium-dependent phospholipid-binding protein that has been associated with tumor suppressive activity in breast cancer, with higher expression correlating with better clinical outcomes and reduced proliferation \citep{anxa7_breast_prognosis, anxa7_tumor_suppressor}. The opposing directions of these effects are consistent with their inferred roles in cancer biology and reinforce the biological plausibility of the gene-level signals identified by the penalized survival model.

\begin{figure}[t]
\centering
\includegraphics[width=0.95\linewidth]{figure2_top_coeffs.png}
\caption{Top elastic net Cox coefficients (log hazard ratios) among selected genes. Positive coefficients indicate higher relapse risk and negative coefficients indicate lower risk.}
\label{fig:coeffs}
\end{figure}

\section{4 Discussion}\label{sec4}

In this study, we applied elastic net–regularized Cox proportional hazards modeling to high-dimensional breast cancer gene expression data to derive a genomic risk score associated with relapse-free survival. By combining regularization with survival analysis, we were able to summarize thousands of correlated expression measurements into a single patient-level score that meaningfully stratified outcomes. The observed separation between high- and low-risk groups, together with a favorable concordance index, highlights the practical utility of penalized survival models for prognostic analysis in settings where the number of predictors far exceeds the number of samples.

A key strength of this approach is the balance it achieves between predictive performance and interpretability. Elastic net regularization selected a moderate subset of genes with nonzero coefficients, avoiding the instability of unpenalized Cox models while retaining biologically plausible contributors to relapse risk. Nonetheless, several limitations should be acknowledged. The analysis was conducted on a single publicly available cohort, and external validation would be necessary to assess generalizability. In addition, the use of a median risk split for visualization may not reflect optimal clinical decision thresholds. Despite these limitations, this work demonstrates a reproducible framework for integrating gene expression data with time-to-event outcomes using modern regularization techniques.

%\section{Competing interests}
%No competing interest is declared.

\section{Appendix}\label{sec:appendix}
\setcounter{figure}{0}
\renewcommand{\thefigure}{S\arabic{figure}}

\begin{figure}[t]
\centering
\includegraphics[width=0.95\linewidth]{supp_pca.png}
\caption{Principal component analysis of the GSE7390 breast cancer expression data using the top 2{,}000 most variable probes, colored by relapse-free survival status.}
\label{fig:pca}
\end{figure}

\subsection{Principal component analysis}
Appendix Fig.~\ref{fig:pca} shows a PCA of the top 2{,}000 most variable probes after normalization, with samples colored by relapse status. As expected, unsupervised variance structure does not yield clear separation by outcome, motivating the use of supervised survival modeling.

\begin{figure}[t]
\centering
\includegraphics[width=0.95\linewidth]{supp_cv_plot.png}
\caption{Cross-validation curve for the elastic net Cox model ($\alpha = 0.5$). The dashed vertical line indicates the selected regularization parameter used for downstream risk score estimation.}
\label{fig:cv}
\end{figure}

\subsection{Elastic net cross-validation}
Appendix Fig.~\ref{fig:cv} displays the cross-validated partial likelihood deviance for the elastic net Cox model. The selected regularization parameter corresponds to a stable, non-degenerate solution and is used for downstream risk score estimation.

%\bibliographystyle{plain}
%\bibliography{reference}
%USE THE BELOW OPTIONS IN CASE YOU NEED AUTHOR YEAR FORMAT.
\bibliographystyle{abbrvnat}
\bibliography{reference}

\section{Attribution of Sources}

External resources were used only to support the understanding of methodological concepts and software tools. In particular, ChatGPT was consulted to clarify the statistical methodology underlying several R packages used in the analysis (e.g. survival modeling and regularization) in order to accurately and appropriately describe these methods in the written Methods section. All data processing, model implementation, plotting, and analysis code were written independently by the author. The structure and interpretation of the results, as well as all figures and manuscript text, were produced by the author. The overall analysis was inspired by prior published work associated with the dataset but was implemented and written independently.

\end{document}
